\documentclass[12pt,spanish]{article}
\usepackage[spanish]{babel}
\usepackage{graphicx}
\usepackage{color}
\usepackage{xcolor}
\usepackage{colortbl}
\usepackage{amsthm,thmtools}
\usepackage{multirow}
\usepackage{amsmath}
\usepackage{subcaption}
\usepackage{adjustbox}
\usepackage{multirow}
\usepackage[hidelinks]{hyperref}
\usepackage{caption}
\usepackage{amsthm}
\usepackage{multicol}
\usepackage{float}
\usepackage{amsfonts}
\usepackage{titling}
\usepackage{soul}
\usepackage{listings}
\usepackage{mathtools}
\usepackage{array}
\usepackage[framemethod=tikz]{mdframed}

\graphicspath{ {../img/}}
\selectlanguage{spanish}
\usepackage[utf8]{inputenc}
\usepackage{graphicx}
\usepackage[a4paper,left=3cm,right=2cm,top=2.5cm,bottom=2.5cm]{geometry}

\newenvironment{solution}{
	\par
	\textbf{Solución}
	\par
	\begin{center}
}
{
	\end{center}
}


\title{Ingeniería de Servidores}
\setlength{\droptitle}{10em}
\author{Carlos Sánchez Páez}

\makeindex
\begin{document}
\definecolor{light-gray}{gray}{0.95}
\lstset{columns=fullflexible,basicstyle=\ttfamily}
\surroundwithmdframed[
  hidealllines=true,
  backgroundcolor=light-gray,
  innerleftmargin=0pt,
  innertopmargin=0pt,
  innerbottommargin=0pt]{lstlisting}


\begin{titlepage}

 \newlength{\centeroffset}
 \setlength{\centeroffset}{-0.5\oddsidemargin}
 \addtolength{\centeroffset}{0.5\evensidemargin}
 \thispagestyle{empty}

 \noindent\hspace*{\centeroffset}
 \begin{minipage}{\textwidth}

  \centering
  \includegraphics[width=0.9\textwidth]{logo_ugr.jpg}\\[1.4cm]

  \textsc{ \Large Modelos de Computación\\[0.2cm]}
  \textsc{GRADO EN INGENIERÍA INFORMÁTICA}\\[1cm]

  {\Huge\bfseries Preguntas de examen resueltas\\}
 \end{minipage}

 \vspace{1.5cm}
 \noindent\hspace*{\centeroffset}
 \begin{minipage}{\textwidth}
  \centering

  \textbf{Autor}\\ {Carlos Sánchez Páez}\\[4ex]
  \includegraphics[width=0.4\textwidth]{etsiit_logo.png}\\[0.1cm]
  \vspace{1.5cm}
  \includegraphics[width=0.5\textwidth]{decsai.jpg}\\[0.1cm]
  \vspace{1cm}
  \textsc{Escuela Técnica Superior de Ingenierías Informática y de Telecomunicación}\\
  \vspace{1cm}
  \textsc{Curso 2019-2020}
 \end{minipage}
\end{titlepage}
\thispagestyle{empty}
\newpage
\tableofcontents{}
\newpage
\listoffigures
\thispagestyle{empty}
\newpage

\section{Tema 1}

\begin{enumerate}
	\item Determinar si la gramática $G=({S,A,B},{a,b,c,d},P,S)$ donde P es el conjunto de reglas de producción:
	\[
		S \implies AB ; A \implies Ab ; A \implies a ; B \implies cB ; B \implies d
	\]
	genera un lenguaje de tipo 3.
	\begin{solution}
		Comenzamos a generar:
		\begin{gather*}
			S \implies AB \xRightarrow[A \implies Ab]{} AbB \xRightarrow[A \implies Ab]{} AbbB \xRightarrow[...]{} Ab^iB \xRightarrow[B \implies cB]{} Ab^icB \\ \xRightarrow[...]{} Ab^ic^jB \xRightarrow[A \implies a]{} ab^ic^jB \xRightarrow[B \implies d]{} ab^ic^jd
		\end{gather*}
		Vemos que generamos el lenguaje ${ab^ic^jd}$, que también se puede generar mediante la siguiente gramática:
		\[
			S \implies aB ; B \implies bB ; B \implies C ; C \implies cC ; C \implies d
		\]
		Como la gramática es de tipo 3 (sólo hay como máximo una variable a la derecha en todas las producciones), el lenguaje también lo es.
	\end{solution}

	\item Diseñar una máquina de estados que calcule el complemento a dos de un número binario.

	\begin{solution}
	El complemento a dos de un número binario se calcula obteniendo su complemento a uno y sumándole uno. Veamos algunos ejemplos:
	\begin{itemize}
		\item $C_2$(\textcolor{red}{1}\textcolor{blue}{100})=$C_1$(1100)+1=0011+1=\textcolor{red}{0}\textcolor{blue}{100}
		\item $C_2$(\textcolor{red}{11}\textcolor{blue}{10})=$C_1$(1110)+1=0001+1=\textcolor{red}{00}\textcolor{blue}{10}
		\item $C_2$(\textcolor{red}{11101}\textcolor{blue}{100})=$C_1$(11101100)+1=00010011+1=\textcolor{red}{00010}\textcolor{blue}{100}
	\end{itemize}
	Tras realizar varias operaciones nos damos cuenta de que existe una codificación que se mantiene:
	\begin{enumerate}
		\item Comenzamos leyendo el número de derecha a izquierda y escribimos lo que leemos en la salida (también de derecha a izquierda).
		\item Cuando encontremos el primer 1 lo escribimos en la cinta y a partir de ahí escribimos el complemento a uno del número que leamos (cambiamos 0 por 1 y viceversa).
	\end{enumerate}
	\newpage
	Esta codificación se puede expresar mediante una máquina de \textit{Mealy}:
	\begin{itemize}
		\item La cabeza lectora y escritora se desplazará de derecha a izquierda.
		\item Estados
			\begin{itemize}
				\item $q_0$: todavía no he leído el primer 1. Si leo 0, escribo 0 y me mantengo. Si leo 1, paso al estado $q_1$ y escribo 1.
				\item $q_1$: ya he leído el primer 1. Ahora debo aplicar el complemento a 1 (si leo 1 escribo 0 y viceversa). En ambos casos me mantengo.
			\end{itemize}
		Es decir, la máquina sería la siguiente:
			\begin{figure}[H]
				\centering
				\includegraphics{ej2.png}
			\end{figure}
	\end{itemize}

	\end{solution}
\end{enumerate}


\end{document}
